\documentclass{article}
\title{ME 202 - Strength of Materials}
\author{Vishal Neeli}
\date{}

\usepackage[a4paper, total={6in, 11in}]{geometry}
\usepackage{textcomp}
\usepackage{hyperref}
\usepackage{amsmath}
\usepackage{xcolor}
\usepackage{grffile}
\usepackage{physics}
\hypersetup{
	colorlinks=true,
	urlcolor=blue,
	linkcolor=cyan,
	filecolor=red
}
\usepackage{amsfonts}

% FOR CODE
% \usepackage{listings}
% \usepackage{color}

% \definecolor{dkgreen}{rgb}{0,0.6,0}
% \definecolor{gray}{rgb}{0.5,0.5,0.5}
% \definecolor{mauve}{rgb}{0.58,0,0.82}

% \lstset{frame=tb,
%   language=Java,
%   aboveskip=3mm,
%   belowskip=3mm,
%   showstringspaces=false,
%   columns=flexible,
%   basicstyle={\small\ttfamily},
%   numbers=none,
%   numberstyle=\tiny\color{gray},
%   keywordstyle=\color{blue},
%   commentstyle=\color{dkgreen},
%   stringstyle=\color{mauve},
%   breaklines=true,
%   breakatwhitespace=true,
%   tabsize=3
% }

\begin{document}
\maketitle

\section{Review}
	\subsection{Strain Energy}
	\begin{align*}
		u = \frac{1}{2} E \epsilon^2 = \frac{1}{2} \sigma \epsilon = \frac{1}{2} \frac{\sigma^2}{E}
	\end{align*}

	\subsection{Poison's Ratio}

	$\epsilon_{axial}= \frac{\delta}{L}$, $\delta$ is the change in longitudinal length\\
	$\epsilon_{radial}= \frac{\delta'}{r}$, $\delta'$ is the change in radius\\

	Poisson's ratio -
	\[\nu = -\frac{\epsilon_{radial}}{\epsilon_{axial}}\]
	\[0\leq \nu \leq 0.5\]

	Shear stress and strain
	\[\tau = G \gamma\]
	Relation between G, E and $\nu$ - 
	\[G= \frac{E}{2(1+\nu)}\]

	For a axial load, 
	\[d\delta = \frac{N(x) dx}{A(x) E(x)}\]


\section{Torsion}
	\subsection{Uniform Torsion}
	\begin{itemize}
		\item \textbf{Angle of twist} : It is the angle by which one end of the rod is displaced wrt the other end of the rod under the effect of some torsion (twisting).
		\item Consider a small cylidrical section which has length $dx$, one end has been displaced wrt another by an angle of $d\phi$. Consider a line $ab$ on the circumference along the length of the cylinder which has be changed to $ab'$. Then shear strain 
		\begin{equation*}
			\gamma_{max} = \frac{bb'}{ab} \qquad (\text{bb' can be assumed to be a straight line})\\
		\end{equation*}
			\[\boxed{\gamma_{max} = \frac{r d\phi}{dx}}\]


		\item Rate of twist or angle of twist per unit length
		\[\theta = \dv{\phi}{x}\]
		\[\gamma_{max} = r \theta\]
		If $\theta$ is constant, then
		\[\gamma_{max}= \frac{r \phi}{L}\] 
		This is called as $\gamma_{max}$ because we are measuring the shear strain at the outer end i.e, with maximum radius and hence, maximum shear strain.
		\[\boxed{\gamma = \rho \theta = \rho \dv{\phi}{x}}\]
		where $\rho$ is the perpendicular distance of the point from the axis (radius) we are considering.
		\[\boxed{\gamma = \frac{\rho}{r} \gamma_{max} }\]

	\end{itemize}
	\subsection{Hooke's Law}
		Hooke' law for shear stress and shear strain
		\begin{align*}
			\tau = G \gamma
				 = G \frac{\rho}{r}\gamma_{max}
				 = \frac{\rho}{r}\tau_{max}
		\end{align*}
		\[\tau_{max} = G \gamma_{max}\]


	\subsection{Torsion Formula}
	\begin{itemize}
		\item Polar moment of inertia (this is integral over area - double integral)-
		\[\boxed{I_P = \int_A \rho^2 dA}\]
		where $\rho$ is the distance at which area element $dA$ is located.
		\item For a circle of radius r, 
		\[I_P = \int_A \rho^2 dA = \int_{\theta = 0}^{2\pi}\int_{\rho = 0}^r \rho^2 (\rho d\theta d\rho) = \frac{\pi r^4}{2} \]
		\[I_{Pcircle} = \frac{\pi r^4}{2}\]
		\item Consider a cross-section of any shape, we are trying to sum all the small torques and equate it to the torque applied on this
		\begin{align*}
			T = \int_A dM &= \int_A \rho \tau dA\\
						  &= \int_A \frac{\rho^2}{r}\tau_{max} dA\\
						  &= \frac{\tau_{max}}{r} \int_A \rho^2 dA\\
						T &= \frac{\tau_{max}}{r} I_P
		\end{align*}
		Or,
		\[\tau_{max} = \frac{Tr}{I_P}\]
		item The shear stress at distance $\rho$ from the center of the bar with polar moment of inertia $I_P$ is
		\[\boxed{\tau = \frac{T\rho}{I_P}}\]
		For a rod of circular cross section of radius r,
		\[\boxed{\tau_{max} = \frac{2T}{\pi r^3} = \frac{16T}{\pi d^3 }}\]

		\item Rate of twist
		\[\boxed{\theta = \frac{T}{GI_P}}\]
		Hence, $GI_P$ is also known as \textbf{Torsional Rigidity}
		\item For a bar in pure torsion ($\theta = const$), the total angle of twist
		\[\boxed{\phi = \frac{TL}{GI_P}}\]
		The quantity $\frac{G I_P}{L}$ is also known as \textbf{torsional stiffness} of the bar.

		\item For a thin circular tube, $I_P = 2 \pi r^3 t$
	\end{itemize}

	\subsection{Non-uniform Torsion}
		For a bar with non-uniform cross-section, tension  :
		\[\boxed{\phi = \int_0^L \frac{T(x) dx}{G I_P(x)}}\]

\section{Hooke's Law}

	\subsection{Stresses in a Plane and Mohr's Circle}
	Using physical equations or using cauchy's stress tensor with rotation matrix, we get
	\begin{gather}
		\sigma_{x'} = \frac{\sigma_x + \sigma_y}{2} + \frac{\sigma_x - \sigma_y}{2} cos2\theta + \tau_{xy} sin2\theta \\
		\tau_{xy'} = -\frac{\sigma_x - \sigma_y}{2} sin2\theta + \tau_{xy} cos2\theta \\
		\sigma_{y'} = \frac{\sigma_x + \sigma_y}{2} - \frac{\sigma_x - \sigma_y}{2} cos2\theta - \tau_{xy} sin2\theta
	\end{gather}

	The above equations can be viewed as parametric equations to a circle.

	On squaring and adding, we get Mohr's circle

	\[\left(\sigma_{x'} -\frac{\sigma_x+\sigma_y}{2} \right)^2 + \tau_{xy'}^2 = \left(\frac{\sigma_x - \sigma_y}{2}\right)^2 + \tau_{xy}^2 \] 

	Center = $(\frac{\sigma_x+\sigma_y}{2},0)$ and 
	Radius = $\sqrt{\left(\frac{\sigma_x - \sigma_y}{2}\right)^2 + \tau_{xy}^2}$\\

	\begin{itemize}
		\item This is plotted by taking $\sigma_{x'}$ on x axis and $\tau{xy'}$ on \textbf{negative} y-axis and angle $2\theta$ anticlockwise as positive.\\
		\item $\tau{xy}$ is taken as positive if it tends to rotate in the anticlockwise direction, negative otherwise.
		\item $\sigma_x$ is taken as positive if it is tensile and negative for compressive.
		\item A rotation of angle $\theta$ in the plane corresponds to a rotation of $2\theta$ on the Mohr's circle.
	\end{itemize}


	\subsection{Hooke's law for plane stresses }
	For a point (isotropic material) under planar stresses ($\sigma_{xz}=\sigma_{yz}=\sigma_{zz} =0$), we have
	\begin{gather*}
		\epsilon_x = \frac{1}{E} (\sigma_x - \nu \sigma_y)\\
		\epsilon_y = \frac{1}{E} (\sigma_y - \nu \sigma_x)\\
		\gamma_{xy} = \frac{\tau_{xy}}{G}\\
		% \left{\text{Using these,}}
		\sigma_x = \frac{E}{1-\nu^2} (1+\nu \epsilon_x)\\
		\sigma_y = \frac{E}{1-\nu^2} (1+\nu \epsilon_y)\\
		\tau_{xy} = G \gamma_{xy}
	\end{gather*}



	\subsection{Volume change and Strain-energy density}
		Let a cuboid of sides a,b and c be under stresses. 
		\[V_0 = abc\]
		\begin{align*}
		V &= (a+a\epsilon_x) (b+b\epsilon_y)  (c+c\epsilon_z)\\
		  &= abc(1+\epsilon_x) (1+\epsilon_y)  (1+\epsilon_z)\\ 
		  &= V_0 (1+\epsilon_x +\epsilon_y + \epsilon_z ) &\text{ignoring the $\epsilon_x\epsilon_y$ terms}
		\end{align*}

		Unit volume change e, also known as \textbf{dialatation}\\
		\[e=\frac{\Delta V}{V_0} = (1+\epsilon_x) (1+\epsilon_y)  (1+\epsilon_z)\]

		Strain-energy density in plane stress,
		\begin{align*}
			u &= \frac{1}{2}(\sigma_x\epsilon_x + \sigma_y \sigma_y + \tau_{xy} \gamma_{xy})\\
			u &= \frac{1}{2E} (\sigma_x^2 +\sigma_y^2 - 2 \nu \sigma_x \sigma_y +\frac{\tau_{xy}^2}{2G})\\
			u &= \frac{E}{2(1-\nu^2)}(\epsilon_x^2 +\epsilon_y^2 +2\nu\epsilon_x \epsilon_y)+\frac{G\gamma_{xy}^2}{2}
		\end{align*}
	\subsection{Hooke's law for triaxial stresses}
	($\sigma_z=\tau_{xz}=\tau_{yz}=0$)
	\begin{gather*}
		\epsilon_x = \frac{1}{E}(\sigma_x -\nu \sigma_y -\nu \sigma_z)\\
		\sigma_x = \frac{E}{(1+\nu)(1-2\nu)}[(1-\nu)\epsilon_x + \nu (\epsilon_y+\epsilon_z)]
	\end{gather*}


	\subsection{Strain Energy in Torsion}
	\begin{gather*}
		d\phi = \frac{T dx}{GJ}
	\end{gather*}
	\begin{align*}
		\text{Energy stored due to torsion, } U &= \int_0^L \frac{1}{2} T d\phi\\
											    &= \int_0^L \frac{T^2 dx}{2GJ} \\
	\end{align*}


\section{Torsion formula}
	\subsection{Torsion Formula for non-prismatic bars}
	\begin{itemize}
	\item For elliptical cross section, maximum shear stress
		\[\tau_{max} = \frac{2T}{\pi ab^2}\]
		\[\phi = \frac{TL}{GJ_e}\]
		\[J_e = \frac{\pi a^3 b^3}{a^2 + b^2}\]


	\item For triangular cross section
		\[\tau_{max} = \frac{T \frac{h}{2}}{J_t}\]
		\[\phi = \frac{TL}{GJ_t}\]
		\[J_e = \frac{h_t}{15\sqrt{3}}\]

	\item For rectangular cross section
		\[\tau_{max} = \frac{T}{k_1 b t^2}\]
		\[\phi = \frac{TL}{(k_2bt^3)G}= \frac{TL}{J_rG}\]
		\[J_r = k_2 b t^3\]
		$k_1$ and $k_2$ are emphirically determined and are dependent on $\frac{b}{t}$

	\item Thin walled open cross sections
	We treat flange and web as seperate rectangles.
		\[J = J_w + 2J_f\]
		\begin{gather*}
		J_f = k_1 b_f t_f^3 \\
		J_w = k_1 (b_w-2t_f) t_w^3\\
		\tau_max = \frac{2T (\frac{t}{2})}{J}\\
		\phi = \frac{TL}{GJ}
		\end{gather*}
	\end{itemize}

	\subsection{Thin walled tubes}
		Consider a small element of length of $dx$, then shear stresses are $\tau_a$,$\tau_b$, $\tau_c$ and $\tau_d$. Then the shear stresses on opposite wall should be equal (to satisfy Newton's second law). Hence,
		\begin{align*}
			F_b &= F_c\\
			\tau_b t_b dx &= \tau_c t_c dx\\
			\tau_b t_b &= \tau_c t_c
		\end{align*}

		Shear flow f,
		\[\boxed{f = \tau t = const}\]

	Deriving torsion formula for the thin walled tubes- 

	\begin{align*}
		T &= \int_0^{L_m} \tau dA \rho\\
		  &= \int_0^{L_m} frds\\
		  &= 2 f A_m 	& \text{where $A_m$ is the area enclosed by the median line.}
	\end{align*}

	\[\boxed{\tau= \frac{T}{2tA_m}}\]

	Using shear stress, we get $J$ -\\
	Shear energy density of a solid under pure shear stress is $\frac{\tau^2}{2G}$. 
	\begin{align*}
		\text{Total shear energy }U &= \int\int \frac{\tau^2}{2G} t dx ds\\
								   &= \int \int \frac{f^2}{2Gt} dx ds\\
								   &= \frac{f^2}{2G} \int_0^L dx \int_0^{L_m} ds\\
								   &= \frac{T^2 L}{2G A_m^2} \int_0^{L_m} ds
	\end{align*}
	Also, $U = \frac{T^2 L}{2G J}$.
	Hence, 
		\[\boxed{J = \frac{4A_m^2}{\int_0^{L_m}\frac{dS}{t}}}\]
	For tube with \textbf{constant thickness},
		\[\boxed{J = \frac{4A_m^2t}{L_m}}\]

	For circular tube,
		\[J= 2\pi r^3 t\]

	For rectangular tube, 
		\[J = \frac{2b^2 h^2 t_1 t_2}{bt_1 + ht_2}\]\\

	Angle of twist,
	\begin{align*}
		\phi &= \frac{TL}{GJ}\\
			 &= \frac{TL}{4GA_m^2}\int_0^{L_m} \frac{ds}{t}
	\end{align*}

	\[\boxed{\phi = \frac{TL}{4GA_m^2}\int_0^{L_m} \frac{ds}{t}}\]











	% \end{itemize}


























% \item Non uniform Torsion
% 	\item For a rod of length on which multiple torsions are applied, angle of twist can be found as
% 	\[\phi= \phi_1 + \phi_2 + \hdots + \phi_n\]
% 	where $\phi_1$ is the angle of twist in 1 part of rod, $\phi_2$ is the angle of twist in 2nd part...

% 	\item For bar with continuously varying cross section,
% 	\[d\phi = \frac{num}{den}\]

% \item Stresses on inclined plane
% 	-




























\end{document}