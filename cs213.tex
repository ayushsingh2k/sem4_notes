\documentclass{article}
\title{}
\author{Vishal Neeli}

\usepackage[a4paper, total={6in, 11in}]{geometry}
\usepackage{textcomp}
\usepackage{hyperref}
\usepackage{amsmath}
\usepackage{xcolor}
\usepackage{grffile}
\usepackage{physics}
\hypersetup{
	colorlinks=true,
	urlcolor=blue,
	linkcolor=cyan,
	filecolor=red
}
\usepackage{amsfonts}

% FOR CODE
% \usepackage{listings}
% \usepackage{color}

% \definecolor{dkgreen}{rgb}{0,0.6,0}
% \definecolor{gray}{rgb}{0.5,0.5,0.5}
% \definecolor{mauve}{rgb}{0.58,0,0.82}

% \lstset{frame=tb,
%   language=Java,
%   aboveskip=3mm,
%   belowskip=3mm,
%   showstringspaces=false,
%   columns=flexible,
%   basicstyle={\small\ttfamily},
%   numbers=none,
%   numberstyle=\tiny\color{gray},
%   keywordstyle=\color{blue},
%   commentstyle=\color{dkgreen},
%   stringstyle=\color{mauve},
%   breaklines=true,
%   breakatwhitespace=true,
%   tabsize=3
% }

\begin{document}
\maketitle

\section{Recursion}

\subsection{Design principle}
\begin{enumerate}
	\item Reduce the problem of size n to a size n-1
	\item Figure out the base case (usually n=0 or 1)
	\item Terminate recursion at the base case. Make sure that every n reaches the base case.

\end{enumerate}

\subsection{Parameterization}
\begin{itemize}
	\item It is consumes extra memory if pass a new array (part of old array copied into this) for recursion. Instead we should pass the indices as the parameters.
	\item In creating recursive methods, it is often useful to define additional functions(or methods) to facilitate recursion.

\end{itemize}























\end{document}